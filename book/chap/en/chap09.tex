In this section we will write a multithreaded, synchronized
program in C.  Appendix~\ref{ccleanup} provides some of the utility
code I use to make the C code a little more palatable.  The
examples in this section depend on that code.

\section{Mutual exclusion}

We'll start by defining a structure that contains shared
variables:

\begin{lstbox}{}
typedef struct {
  int counter;
  int end;
  int *array;
} Shared;

Shared *make_shared (int end)
{
  int i;
  Shared *shared = check_malloc (sizeof (Shared));

  shared->counter = 0;
  shared->end = end;

  shared->array = check_malloc (shared->end * sizeof(int));
  for (i=0; i<shared->end; i++) {
    shared->array[i] = 0;
  }
  return shared;
}
\end{lstbox}

{\tt counter} is a shared variable that will be incremented by
concurrent threads until it reaches {\tt end}.  We will use
{\tt array} to check for synchronization errors by keeping track
of the value of {\tt counter} after each increment.

\subsec{Parent code}

Here is the code the parent thread uses to create threads
and wait for them to complete:

\begin{lstbox}{}
int main ()
{
  int i;
  pthread_t child[NUM_CHILDREN];

  Shared *shared = make_shared (100000);

  for (i=0; i<NUM_CHILDREN; i++) {
    child[i] = make_thread (entry, shared);
  }

  for (i=0; i<NUM_CHILDREN; i++) {
    join_thread (child[i]);
  }

  check_array (shared);
  return 0;
}
\end{lstbox}

The first loop creates the child threads; the second loop waits
for them to complete.  When the last child has finished, the parent
invokes {\tt check\_array} to check for errors.
{\tt make\_thread} and {\tt join\_thread} are defined in
Appendix~\ref{ccleanup}.

\subsec{Child code}

Here is the code that is executed by each of the children:

\begin{lstbox}{}
void child_code (Shared *shared)
{
  while (1) {
    if (shared->counter >= shared->end) {
      return;
    }
    shared->array[shared->counter]++;
    shared->counter++;
  }
}
\end{lstbox}

Each time through the loop, the child threads use {\tt counter}
as an index into {\tt array} and increment the corresponding element.
Then they increment {\tt counter} and check to see if they're done.

\subsec{Synchronization errors}

If everything works correctly, each element of the array should be
incremented once.  So to check for errors, we can just count the
number of elements that are not 1:

\begin{lstbox}{}
void check_array (Shared *shared)
{
  int i, errors=0;

  for (i=0; i<shared->end; i++) {
    if (shared->array[i] != 1) errors++;
  }
  printf ("%d errors.\n", errors);
}
\end{lstbox}

You can download this program (including the cleanup code) from
\url{greenteapress.com/semaphores/counter.c}

If you compile and run the program, you should see output like this:

\begin{verbatim}
Starting child at counter 0
10000
20000
30000
40000
50000
60000
70000
80000
90000
Child done.
Starting child at counter 100000
Child done.
Checking...
0 errors.
\end{verbatim}

Of course, the interaction of the children depends on details
of your operating system and also other programs running on your
computer.  In the example shown here, one thread ran all the way
from 0 to {\tt end} before the other thread got started, so it is
not surprising that there were no synchronization errors.

But as {\tt end} gets bigger, there are more context switches between
the children.  On my system I start to see errors when
{\tt end} is 100,000,000.

Puzzle: use semaphores to enforce exclusive access to the shared
variables and run the program again to confirm that there are
no errors.

%% \clearemptydoublepage
\subsec{Mutual exclusion hint}

Here is the version of {\tt Shared} I used in my solution:

\begin{lstbox}{}
typedef struct {
  int counter;
  int end;
  int *array;
  Semaphore *mutex;  (*\label{declaremutex}*)
} Shared;

Shared *make_shared (int end)
{
  int i;
  Shared *shared = check_malloc (sizeof (Shared));

  shared->counter = 0;
  shared->end = end;

  shared->array = check_malloc (shared->end * sizeof(int));
  for (i=0; i<shared->end; i++) {
    shared->array[i] = 0;
  }
  shared->mutex = make_semaphore(1);  (*\label{initmutex}*)
  return shared;
}
\end{lstbox}

Line~\ref{declaremutex} declares {\tt mutex} as a Semaphore;
Line~\ref{initmutex} initializes the mutex with the value 1.


%% \clearemptydoublepage
\subsec{Mutual exclusion solution}

Here is the synchronized version of the child code:

\begin{lstbox}{}
void child_code (Shared *shared)
{
  while (1) {
    sem_wait(shared->mutex);
    if (shared->counter >= shared->end) {
      sem_signal(shared->mutex);
      return;
    }

    shared->array[shared->counter]++;
    shared->counter++;
    sem_signal(shared->mutex);
  }
}
\end{lstbox}

There is nothing too surprising here; the only tricky thing
is to remember to release the mutex before the {\tt return}
statement.

You can download this solution from 
\url{greenteapress.com/semaphores/counter_mutex.c}


%% \clearemptydoublepage
\blankpage
\section{Make your own semaphores}
\label{makeyourown}

The most commonly used synchronization tools for programs that use
Pthreads are mutexes and condition variables, not semaphores.  For an
explanation of these tools, I recommend Butenhof's {\em Programming
with POSIX Threads} \cite{butenhof}.

Puzzle: read about mutexes and condition variables, and then
use them to write an implementation of semaphores.

You might want to use the following utility code in your solutions.
Here is my wrapper for Pthreads mutexes:

\begin{lstbox}{}
typedef pthread_mutex_t Mutex;

Mutex *make_mutex ()
{
  Mutex *mutex = check_malloc (sizeof(Mutex));
  int n = pthread_mutex_init (mutex, NULL);
  if (n != 0) perror_exit ("make_lock failed"); 
  return mutex;
}

void mutex_lock (Mutex *mutex)
{
  int n = pthread_mutex_lock (mutex);
  if (n != 0) perror_exit ("lock failed");
}

void mutex_unlock (Mutex *mutex)
{
  int n = pthread_mutex_unlock (mutex);
  if (n != 0) perror_exit ("unlock failed");
}
\end{lstbox}

%% \newpage
And my wrapper for Pthread condition variables:

\begin{lstbox}{}
typedef pthread_cond_t Cond;

Cond *make_cond ()
{
  Cond *cond = check_malloc (sizeof(Cond)); 
  int n = pthread_cond_init (cond, NULL);
  if (n != 0) perror_exit ("make_cond failed");
  return cond;
}

void cond_wait (Cond *cond, Mutex *mutex)
{
  int n = pthread_cond_wait (cond, mutex);
  if (n != 0) perror_exit ("cond_wait failed");
}

void cond_signal (Cond *cond)
{
  int n = pthread_cond_signal (cond);
  if (n != 0) perror_exit ("cond_signal failed");
}
\end{lstbox}



%% \clearemptydoublepage
\subsec{Semaphore implementation hint}

Here is the structure definition I used for my semaphores:

\begin{lstbox}{}
typedef struct {
  int value, wakeups;
  Mutex *mutex;
  Cond *cond;
} Semaphore;
\end{lstbox}

{\tt value} is the value of the semaphore.  {\tt wakeups} counts
the number of pending signals; that is, the number of threads
that have been woken but have not yet resumed execution.  The reason
for wakeups is to make sure that our semaphores have
Property 3, described in Section~\ref{props}.

{\tt mutex} provides exclusive access to {\tt value} and
{\tt wakeups}; {\tt cond} is the condition variable threads
wait on if they wait on the semaphore.

Here is the initialization code for this structure:

\begin{lstbox}{}
Semaphore *make_semaphore (int value)
{
  Semaphore *semaphore = check_malloc (sizeof(Semaphore));
  semaphore->value = value;
  semaphore->wakeups = 0;
  semaphore->mutex = make_mutex ();
  semaphore->cond = make_cond ();
  return semaphore;
}
\end{lstbox}


%% \clearemptydoublepage
\subsec{Semaphore implementation}

Here is my implementation of semaphores using Pthread's mutexes
and condition variables:

\begin{lstbox}{}
void sem_wait (Semaphore *semaphore)
{
  mutex_lock (semaphore->mutex);                 (*\label{sementer}*)
  semaphore->value--;

  if (semaphore->value < 0) {
    do {                                                (*\label{dowhile}*)
      cond_wait (semaphore->cond, semaphore->mutex);
    } while (semaphore->wakeups < 1);
    semaphore->wakeups--;
  }
  mutex_unlock (semaphore->mutex);
}

void sem_signal (Semaphore *semaphore)
{
  mutex_lock (semaphore->mutex);
  semaphore->value++;

  if (semaphore->value <= 0) {
    semaphore->wakeups++;
    cond_signal (semaphore->cond);
  }
  mutex_unlock (semaphore->mutex);
}
\end{lstbox}

Most of this is straightforward; the only thing that might be 
tricky is the {\tt do...while} loop at Line~\ref{dowhile}.
This is an unusual way to use a condition variable, but in
this case it is necessary.

Puzzle: why can't we replace this {\tt do...while} loop
with a {\tt while} loop?

%% \clearemptydoublepage
\subsec{Semaphore implementation detail}

With a {\tt while} loop, this implementation would not have
Property 3.  It would be possible for a thread to signal
and then run around and catch its own signal.

With the {\tt do...while} loop, it is guaranteed\footnote{Well,
  almost.  It turns out that a well-timed spurious wakeup (see
  \url{http://en.wikipedia.org/wiki/Spurious_wakeup}) can violate this
  guarantee.} that when a thread signals, one of the waiting threads
will get the signal, even if another thread gets the mutex at
Line~\ref{sementer} before one of the waiting threads resumes.

