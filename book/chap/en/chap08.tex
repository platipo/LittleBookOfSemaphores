By using pseudocode, we have avoided some of the ugly
details of synchronization in the real world.  In this chapter
we'll look at real synchronization code in Python; in the
next chapter we'll look at C.

Python provides a reasonably pleasant multithreading environment,
complete with Semaphore objects.  It has
a few foibles, but there is some cleanup code in Appendix~\ref{cleanup}
that makes things a little better.

Here is a simple example:

\begin{lstbox}{}
from threading_cleanup import *

class Shared:
    def __init__(self):
        self.counter = 0

def child_code(shared):
    while True:
        shared.counter += 1
        print shared.counter
        time.sleep(0.5)

shared = Shared()
children = [Thread(child_code, shared) for i in range(2)]
for child in children: child.join()
\end{lstbox}

The first line runs the cleanup code from Appendix~\ref{cleanup};
I will leave this line out of the other examples.

{\tt Shared} defines an object type that will contain shared variables.
Global variables are also shared between threads, but we won't
use any in these examples.  Threads that are local in the sense
that they are declared inside a function are also local in the
sense that they are thread-specific.

The child code is an infinite loop that increments {\tt counter},
prints the new value, and then sleeps for 0.5 seconds.

The parent thread creates {\tt shared} and two children,
then waits for the children to exit (which in this case, they won't).

\section{Mutex checker problem}

Diligent students of synchronization will notice that the
children make unsynchronized updates to {\tt counter}, which
is not safe!  If you run this program, you might see some
errors, but you probably won't.  The nasty thing about synchronization
errors is that they are unpredictable, which means that even
extensive testing may not reveal them.

To detect errors, it is often necessary to automate the search.
In this case, we can detect errors by keeping track of the values
of {\tt counter}.

\begin{lstbox}{}
class Shared:
    def __init__(self, end=10):
        self.counter = 0
        self.end = end
        self.array = [0]* self.end

def child_code(shared):
    while True:
        if shared.counter >= shared.end: break
        shared.array[shared.counter] += 1
        shared.counter += 1

shared = Shared(10)
children = [Thread(child_code, shared) for i in range(2)]
for child in children: child.join()
print shared.array
\end{lstbox}

In this example, {\tt shared} contains a list (misleadingly
named {\tt array}) that keeps track of the number of times
each value of counter is used.
Each time through the loop, the children check {\tt counter}
and quit if it exceeds {\tt end}.  If not, they use {\tt counter}
as an index into {\tt array} and increment the corresponding
entry.  Then they increment {\tt counter}.

If everything works correctly, each entry in the array should
be incremented exactly once.  When the children exit, the parent
returns from {\tt join} and prints the value of {\tt array}.
When I ran the program, I got
%
\begin{verbatim}
[1, 1, 1, 1, 1, 1, 1, 1, 1, 1]
\end{verbatim}
%
which is disappointingly correct.  If we increase the size of
the array, we might expect more errors, but it also gets harder
to check the result.

%% \newpage
\vspace{1em}
We can automate the checker by making
a histogram of the results in the array:

\begin{lstbox}{}
class Histogram(dict):
    def __init__(self, seq=[]):
        for item in seq:
            self[item] = self.get(item, 0) + 1

print Histogram(shared.array)
\end{lstbox}

Now when I run the program, I get

\begin{verbatim}
{1: 10}
\end{verbatim}
%
which means that the value 1 appeared 10 times, as expected.  No
errors so far, but if we make {\tt end} bigger, things get more
interesting:

\begin{verbatim}
end = 100, {1: 100}
end = 1000, {1: 1000}
end = 10000, {1: 10000}
end = 100000, {1: 27561, 2: 72439}
\end{verbatim}
%
Oops!  When {\tt end} is big enough that there are a lot of
context switches between the children, we start to get synchronization
errors.  In this case, we get a {\em lot} of errors, which suggests
that the program falls into a recurring pattern where threads are 
consistently interrupted in the critical section.

This example demonstrates one of the dangers of synchronization
errors, which is that they may be rare, but they are not random.
If an error occurs one time in a million, that doesn't mean it
won't happen a million times in a row.

Puzzle: add synchronization code to this program to enforce
exclusive access to the shared variables.  You can download the
code in this section from \url{greenteapress.com/semaphores/counter.py}


%% \clearemptydoublepage
\subsec {Mutex checker hint}

Here is the version of {\tt Shared} I used:

\begin{lstbox}{}
class Shared:
    def __init__(self, end=10):
        self.counter = 0
        self.end = end
        self.array = [0]* self.end
        self.mutex = Semaphore(1)
\end{lstbox}

The only change is the Semaphore named {\tt mutex}, which should
come as no surprise.

%% \clearemptydoublepage
\subsec {Mutex checker solution}

Here is my solution:

\begin{lstbox}{}
def child_code(shared):
    while True:
        shared.mutex.wait()
        if shared.counter < shared.end:
            shared.array[shared.counter] += 1
            shared.counter += 1
            shared.mutex.signal()
        else:
            shared.mutex.signal()
            break
\end{lstbox}

Although this is not the most difficult synchronization problem
in this book, you might have found it tricky to get the details
right.  In particular, it is easy to forget to signal the mutex
before breaking out of the loop, which would cause a deadlock.

I ran this solution with {\tt end = 1000000}, and got the
following result:

\begin{verbatim}
{1: 1000000}
\end{verbatim}

Of course, that doesn't mean my solution is correct, but it is
off to a good start.


%% \clearemptydoublepage
\blankpage
\section {The coke machine problem}

The following program simulates producers and consumers
adding and removing cokes from a coke machine:

\begin{lstbox}{}
import random

class Shared:
    def __init__(self, start=5):
        self.cokes = start

def consume(shared):
    shared.cokes -= 1
    print shared.cokes

def produce(shared):
    shared.cokes += 1
    print shared.cokes

def loop(shared, f, mu=1):
    while True:
        t = random.expovariate(1.0/mu)
        time.sleep(t)
        f(shared)

shared = Shared()
fs = [consume]*2 + [produce]*2 
threads = [Thread(loop, shared, f) for f in fs]
for thread in threads: thread.join()
\end{lstbox}

The capacity is 10 cokes, and that the machine is initially
half full.  So the shared variable {\tt cokes} is 5.

The program creates 4 threads, two producers and two consumers.
They both run {\tt loop}, but producers invoke {\tt produce}
and consumers invoke {\tt consume}.  These functions make
unsynchronized access to a shared variable, which is a no-no.

Each time through the loop, producers and consumers sleep for a
duration chosen from an exponential distribution with mean {\tt mu}.
Since there are two producers and two consumers, two cokes get added
to the machine per second, on average, and two get removed.

So on average the number of cokes is constant, but in the short
run in can vary quite widely.  If you run the program for a
while, you will probably see the value of {\tt cokes} dip
below zero, or climb above 10.  Of course, neither of these
should happen.

Puzzle: add code to this program to enforce the following
synchronization constraints:

\begin{itemize}

\item Access to {\tt cokes} should be mutually exclusive.

\item If the number of cokes is zero, consumers should block
until a coke is added.

\item If the number of cokes is 10, producers should block
until a coke is removed.

\end{itemize}

You can download the program from
\url{greenteapress.com/semaphores/coke.py}


%% \clearemptydoublepage
\subsec {Coke machine hint}

Here are the shared variables I used in my solution:

\begin{lstbox}{}
class Shared:
    def __init__(self, start=5, capacity=10):
        self.cokes = Semaphore(start)
        self.slots = Semaphore(capacity-start)
        self.mutex = Semaphore(1)
\end{lstbox}

{\tt cokes} is a Semaphore now (rather than a simple integer), 
which makes it tricky to print its value.  Of course, you
should never access the value of a Semaphore, and Python in
its usual do-gooder way doesn't provide any of the cheater
methods you see in some implementations.

But you might find it interesting to know that the value
of a Semaphore is stored in a private attribute named
{\tt \_Semaphore\_\_value}.  Also, in case you don't know,
Python doesn't actually enforce any restriction on access to
private attributes.  You should never access it, of course,
but I thought you might be interested.

Ahem.


%% \clearemptydoublepage
\subsec {Coke machine solution}

If you've read the rest of this book, you should have had no
trouble coming up with something at least as good as this:

\begin{lstbox}{}
def consume(shared):
    shared.cokes.wait()
    shared.mutex.wait()
    print shared.cokes.value()
    shared.mutex.signal()
    shared.slots.signal()

def produce(shared):
    shared.slots.wait()
    shared.mutex.wait()
    print shared.cokes._Semaphore__value
    shared.mutex.signal()
    shared.cokes.signal()
\end{lstbox}

If you run this program for a while, you should be able to confirm
that the number of cokes in the machine is never negative or greater
than 10.  So this solution seems to be correct.

So far.

