% Less classical synchronization problems

\section{The dining savages problem}

This problem is from Andrews's 
{\em Concurrent Programming} \cite{andrews}.

\begin {quotation}
A tribe of savages eats communal dinners from a large pot that can
hold M servings of stewed missionary\footnote{This problem is based on
a cartoonish representation of the history of Western missionaries
among hunter-gatherer societies.  Some humor is intended by the
allusion to the Dining Philosophers problem, but the representation of
``savages'' here isn't intended to be any more realistic than the
previous representation of philosophers.  If you are interested in
hunter-gatherer societies, I recommend Jared Diamond's {\em Guns,
Germs and Steel}, Napoleon Chagnon's {\em The Yanomamo}, and Redmond
O'Hanlon's {\em In Trouble Again}, but not Tierney's {\em Darkness in
El Dorado}, which I believe is unreliable.}.  When a savage wants to
eat, he helps himself from the pot, unless it is empty.  If the pot is
empty, the savage wakes up the cook and then waits until the cook has
refilled the pot.
\end{quotation}

Any number of savage threads run the following code:

\begin{lstbox}{Unsynchronized savage code}
while True:
    getServingFromPot()
    eat()
\end{lstbox}

And one cook thread runs this code:

\begin{lstbox}{Unsynchronized cook code}
while True:
    putServingsInPot(M)
\end{lstbox}

The synchronization constraints are:

\begin{itemize}

\item Savages cannot invoke {\tt getServingFromPot} if the
pot is empty.

\item The cook can invoke {\tt putServingsInPot} only if
the pot is empty.

\end{itemize}

\begin{puzzlebox}{Puzzle}
Add code for the savages and the cook that
satisfies the synchronization constraints.


%% \clearemptydoublepage
%%% \subsec{Dining Savages hint}
\tcbsubtitle{Hint}

It is tempting to use a semaphore to keep track of the number of
servings, as in the producer-consumer problem.  But in order to signal
the cook when the pot is empty, a thread would have to know before
decrementing the semaphore whether it would have to wait, and we just
can't do that.

An alternative is to use a scoreboard to
keep track of the number of servings.  If a savage finds
the counter at zero, he wakes the cook and waits for a signal
that the pot is full.  Here are the variables I used:

\begin{lstbox}{Dining Savages hint}
servings = 0
mutex = Semaphore(1)
emptyPot = Semaphore(0)
fullPot = Semaphore(0)
\end{lstbox}

Not surprisingly, {\tt emptyPot} indicates that the pot is empty and
{\tt fullPot} indicates that the pot is full.
\end{puzzlebox}

%% \clearemptydoublepage
%%% \subsec{Dining Savages solution}
\sub{Dining Savages solution}

My solution is a combination of the scoreboard pattern
with a rendezvous.
Here is the code for the cook:

\begin{lstbox}{Dining Savages solution (cook)}
while True:
    emptyPot.wait()
    putServingsInPot(M)
    fullPot.signal()
\end{lstbox}

The code for the savages is only a little more complicated.
As each savage passes through the mutex, he checks the pot.
If it is empty, he signals the cook and waits.  Otherwise,
he decrements {\tt servings} and gets a serving from the pot.

\begin{lstbox}{Dining Savages solution (savage)}
while True:
    mutex.wait()
        if servings == 0:
            emptyPot.signal()
            fullPot.wait()
            servings = M
        servings -= 1
	getServingFromPot()
    mutex.signal()

    eat()
\end{lstbox}

It might seem odd that the savage, rather than the cook, sets
{\tt servings = M}.  That's not really necessary; when the cook
runs {\tt putServingsInPot}, we know that the savage that holds
the mutex is waiting on {\tt fullPot}.  So the cook could
access {\tt servings} safely.  But in this case, I decided to
let the savage do it so that it is clear from looking at the
code that all accesses to {\tt servings} are inside the mutex.

This solution is deadlock-free.  The only opportunity for
deadlock comes when the savage that holds {\tt mutex} waits
for {\tt fullPot}.  While he is waiting, other savages are
queued on {\tt mutex}.  But eventually the cook will run and
signal {\tt fullPot}, which allows the waiting savage
to resume and release the mutex.

Does this solution assume that the pot is thread-safe, or does it
guarantee that {\tt putServingsInPot} and {\tt getServingFromPot}
are executed exclusively?


%% \clearemptydoublepage
\blankpage
\section{The barbershop problem}

The original barbershop problem was proposed by
Dijkstra.  A variation of it appears in 
Silberschatz and Galvin's {\em Operating Systems Concepts}
\cite{silberschatz}.

\begin {quotation}
A barbershop consists of a waiting room with $n$ chairs, and the
barber room containing the barber chair.  If there are no customers to
be served, the barber goes to sleep.  If a customer enters the
barbershop and all chairs are occupied, then the customer leaves the
shop.  If the barber is busy, but chairs are available, then the
customer sits in one of the free chairs.  If the barber is asleep, the
customer wakes up the barber.  Write a program to coordinate the
barber and the customers.
\end{quotation}

To make the problem a little more concrete, I added the
following information:

\begin{itemize}

\item Customer threads should invoke a function named {\tt getHairCut}.

\item If a customer thread arrives when the shop is full, 
it can invoke {\tt balk}, which does not return.

\item The barber thread should invoke {\tt cutHair}.

\item When the barber invokes {\tt cutHair} there should
be exactly one thread invoking {\tt getHairCut} concurrently.

\end{itemize}

\begin{puzzlebox}{Puzzle}
Write a solution that guarantees these constraints.


%% \clearemptydoublepage
%%% \subsec{Barbershop hint}
\tcbsubtitle{Hint}

\lstinputbox{Barbershop hint}{../code/barber.py.0}


{\tt n} is the total number of customers that can be in the shop:
three in the waiting room and one in the chair.

{\tt customers} counts the number of customers in the shop;
it is protected by {\tt mutex}.

The barber waits on {\tt customer} until a customer enters the
shop, then the customer waits on {\tt barber} until the barber
signals him to take a seat.

After the haircut, the customer signals {\tt customerDone} and
waits on {\tt barberDone}.
\end{puzzlebox}

%% \clearemptydoublepage
%%% \subsec{Barbershop solution}
\sub{Barbershop solution}

This solution combines a scoreboard and two rendezvouses\footnote{The
  plural of rendezvous is rare, and not all dictionaries agree about
  what it is.  Another possibility is that the plural is also spelled
  ``rendezvous,'' but the final ``s'' is pronounced.}.  Here is the
code for customers:

\lstinputbox{Barbershop solution (customer)}{../code/barber.py.1}

If there are $n$ customers in the shop any customers that
arrive immediately invoke {\tt balk}.
Otherwise each customer signals {\tt customer} and waits on
{\tt barber}.

Here is the code for barbers:

\lstinputbox{Barbershop solution (barber)}{../code/barber.py.2}

Each time a customer signals,
the barber wakes, signals {\tt barber}, and gives one
hair cut.  If another customer arrives while the barber
is busy, then on the next iteration the barber will pass
the {\tt customer} semaphore without sleeping.

The names for {\tt customer} and {\tt barber} are based on
the naming convention for a rendezvous, so {\tt customer.wait()}
means ``wait for a customer,'' not ``customers wait here.''

The second rendezvous, using {\tt customerDone} and {\tt barberDone},
ensures that the hair cut is done before the barber loops around to
let the next customer into the critical section.

This solution is in \verb"sync_code/barber.py" (see~\ref{sync.py}).


%% \clearemptydoublepage
\blankpage
\section{The FIFO barbershop}

In the previous solution there is no guarantee that customers are
served in the order they arrive.  Up to {\tt n} customers can pass
the turnstile, signal
{\tt customer} and wait on {\tt barber}.  When the barber signal
{\tt barber}, any of the customers might proceed.

\begin{puzzlebox}{Puzzle}
Modify this solution so that customers are served in the order they
pass the turnstile.

Hint: you can refer to the current thread as {\tt self}, so if you
write {\tt self.sem = Semaphore(0)}, each thread gets its own
semaphore.


%% \clearemptydoublepage
%%% \subsec{FIFO barbershop hint}
\tcbsubtitle{Hint}

My solution uses a list of semaphores named {\tt queue}.

\lstinputbox{FIFO barbershop hint}{../code/barber2.py.0}

As each thread passes the turnstile, it creates a thread and puts it
in the queue.

Instead of waiting on {\tt barber}, each thread waits on its own
semaphore.  When the barber wakes up, he removes a thread from the queue and
signals it.
\end{puzzlebox}


%% \clearemptydoublepage
%%% \subsec{FIFO barbershop solution}
\sub{FIFO barbershop solution}

Here is the modified code for customers:

\lstinputbox{FIFO barbershop solution (customer)}{../code/barber2.py.1}

And the code for barbers:

\lstinputbox{FIFO barbershop solution (barber)}{../code/barber2.py.2}

Notice that the barber has to get {\tt mutex} to access the
queue.

This solution is in \verb"sync_code/barber2.py" (see~\ref{sync.py}).


%% \clearemptydoublepage
\blankpage
\section {Hilzer's Barbershop problem}

William Stallings \cite{stallings} presents a more complicated version
of the barbershop problem, which he attributes to Ralph Hilzer at the
California State University at Chico.

\begin{quotation}
Our barbershop has three chairs, three barbers, and a waiting
area that can accommodate four customers on a sofa and that has
standing room for additional customers.  Fire codes limit the
total number of customers in the shop to 20.

A customer will not enter the shop if it is filled to capacity with
other customers.  Once inside, the customer takes a seat on the sofa
or stands if the sofa is filled.  When a barber is free, the customer
that has been on the sofa the longest is served and, if there are any
standing customers, the one that has been in the shop the longest
takes a seat on the sofa.  When a customer's haircut is finished, any
barber can accept payment, but because there is only one cash
register, payment is accepted for one customer at a time.  The barbers
divide their time among cutting hair, accepting payment, and sleeping
in their chair waiting for a customer.
\end{quotation}


In other words, the following synchronization constraints apply:

\begin{itemize}

\item Customers invoke the following functions in order:
{\tt enterShop}, {\tt sitOnSofa},
{\tt getHairCut}, {\tt pay}.

\item Barbers invoke {\tt cutHair} and {\tt acceptPayment}.

\item Customers cannot invoke {\tt enterShop} if the shop
is at capacity.

\item If the sofa is full, an arriving customer cannot invoke 
{\tt sitOnSofa}.

\item When a customer invokes {\tt getHairCut} there should be
a corresponding barber executing {\tt cutHair} concurrently,
and vice versa.

\item It should be possible for up to three customers to execute
{\tt getHairCut} concurrently, and up to three barbers to execute
{\tt cutHair} concurrently.

\item The customer has to {\tt pay} before the barber can
{\tt acceptPayment}.

\item The barber must {\tt acceptPayment} before the customer can
exit.

\end{itemize}

\begin{puzzlebox}{Puzzle}
Write code that enforces the synchronization
constraints for Hilzer's barbershop.


%%% \subsec {Hilzer's barbershop hint}
\tcbsubtitle{Hilzer's barbershop hint}

Here are the variables I used in my solution:

\lstinputbox{Hilzer's barbershop hint}{../code/barber3.py.0}

{\tt mutex} protects {\tt customers}, which keeps track of the
number of customers in the shop, and {\tt queue1} which is a list
of semaphores for threads waiting for a seat on the sofa.

{\tt mutex2} protects {\tt queue2}, which is a list
of semaphores for threads waiting for a chair.

{\tt sofa} is a multiplex that enforces the maximum number of customers
on the sofa.

{\tt customer1} signals that there is a customer in {\tt queue1}, and
{\tt customer2} signals that there is a customer in {\tt queue2}.

{\tt payment} signals that a customer has paid, and {\tt receipt}
sigmals that a barber has accepted payment.
\end{puzzlebox}


%% \clearemptydoublepage
%%% \subsec {Hilzer's barbershop solution}
\sub{Hilzer's barbershop solution}

This solution is considerably more complex than I expected.  I
am not sure if Hilzer had something simpler in mind, but here is the
best I could do.

\lstinputbox{Hilzer's barbershop solution (customer)}{../code/barber3.py.1}

The first paragraph is the same as in the previous solution.  When
a customer arrives, it checks the counter and either balks or adds
itself to the queue.  Then it signals a barber.

When the customer gets out of queue, it enters the multiplex,
sits on the couch and adds itself to the second queue.

When it gets out of {\em that} queue, it gets a haircut, pays,
and then exits.

\lstinputbox{Hilzer's barbershop solution (barber)}{../code/barber3.py.2}

Each barber waits for a customer to enter, signals the customer's
semaphore to get it out of queue, then waits for it to claim a seat
on the sofa.  This enforces the FIFO requirement.

The barber waits for the customer to join the second queue and then
signals it, allowing the customer to claim a chair.

Each barber admits one customer to the chair, so there can by up
to three concurrent haircuts.  Because there is only one cash
register, the customer has to get {\tt mutex}.  The customer
and barber rendezvous at the cash register, then both exit.

This solution satisfies the synchonization constraints, but it leaves
the sofa underutilized.  Because there are only three barbers, there
can never be more than three customers on the sofa, so the multiplex
is unnecessary.

This solution is in \verb"sync_code/barber3.py" (see~\ref{sync.py}).

The only way I can think of to solve this problem is to create a third
kind of thread, which I can an usher.  The ushers manage {\tt queue1}
and the barbers manage {\tt queue2}.  If there are 4 ushers and 3 barbers,
the sofa can be fully utilized.

This solution is in \verb"sync_code/barber4.py" (see~\ref{sync.py}).


%% \clearemptydoublepage
\blankpage
\section{The Santa Claus problem}

This problem is from William Stallings's
{\em Operating Systems} \cite{stallings},
but he attributes it to John Trono of St. Michael's College in
Vermont.

\begin{quotation}
Santa Claus sleeps in his shop at the North Pole and can only be
awakened by either (1) all nine reindeer being back from their
vacation in the South Pacific, or (2) some of the elves having
difficulty making toys; to allow Santa to get some sleep, the elves
can only wake him when three of them have problems.  When three elves
are having their problems solved, any other elves wishing to visit
Santa must wait for those elves to return.  If Santa wakes up to find
three elves waiting at his shop's door, along with the last reindeer
having come back from the tropics, Santa has decided that the elves can
wait until after Christmas, because it is more important to get his
sleigh ready.  (It is assumed that the reindeer do not want to leave
the tropics, and therefore they stay there until the last possible
moment.)  The last reindeer to arrive must get Santa while the others
wait in a warming hut before being harnessed to the sleigh.
\end{quotation}

Here are some addition specifications:

\begin {itemize}

\item After the ninth reindeer arrives, Santa must invoke 
{\tt prepareSleigh}, and then all nine reindeer must
invoke {\tt getHitched}.

\item After the third elf arrives, Santa must invoke {\tt helpElves}.
Concurrently, all three elves should invoke {\tt getHelp}.

\item All three elves must invoke {\tt getHelp} before any additional
elves enter (increment the elf counter).

\end {itemize}

Santa should run in a loop so he can help many sets of elves.
We can assume that there are exactly 9 reindeer, but there may
be any number of elves.  

\begin{puzzlebox}{Hint}
%% \clearemptydoublepage
%%% \subsec {Santa problem hint}

\begin{lstbox}{Santa problem hint}
elves = 0
reindeer = 0
santaSem = Semaphore(0)
reindeerSem = Semaphore(0)
elfTex = Semaphore(1)
mutex = Semaphore(1)
\end{lstbox}

{\tt elves} and {\tt reindeer} are counters, both protected
by {\tt mutex}.  Elves and reindeer get {\tt mutex} to modify the
counters; Santa gets it to check them.

Santa waits on {\tt santaSem} until either an elf or a reindeer
signals him.

The reindeer wait on {\tt reindeerSem} until Santa signals them to
enter the paddock and get hitched.

The elves use {\tt elfTex} to prevent additional elves from
entering while three elves are being helped.
\end{puzzlebox}


%% \clearemptydoublepage
%%% \subsec {Santa problem solution}
\sub{Santa problem solution}

Santa's code is pretty straightforward.  Remember that it
runs in a loop.

\begin{lstbox}{Santa problem solution (Santa)}
santaSem.wait()
mutex.wait()
    if reindeer >= 9:
        prepareSleigh()
	reindeerSem.signal(9)
        reindeer -= 9
    else if elves == 3:
        helpElves()
mutex.signal()
\end{lstbox}

When Santa wakes up, he checks which of the two conditions
holds and either deals with the reindeer or the waiting elves.
If there are nine reindeer waiting,
Santa invokes {\tt prepareSleigh}, then signals {\tt reindeerSem}
nine times, allowing the reindeer to invoke {\tt getHitched}.
If there are elves waiting, Santa just
invokes {\tt helpElves}.  There is no need for the elves to wait
for Santa; once they signal {\tt santaSem}, they can
invoke {\tt getHelp} immediately.

Santa doesn't have to decrement
the {\tt elves} counter because the elves do it on their way
out.

Here is the code for reindeer:

\begin{lstbox}{Santa problem solution (reindeer)}
mutex.wait()
    reindeer += 1
    if reindeer == 9:
        santaSem.signal()
mutex.signal()

reindeerSem.wait()
getHitched()
\end{lstbox}

The ninth reindeer signals Santa and then joins the other
reindeer waiting on {\tt reindeerSem}.  When Santa signals, the
reindeer all execute {\tt getHitched}.

The elf code is similar, except that when the third elf arrives
it has to bar subsequent arrivals until the first three have
executed {\tt getHelp}.

%% \newpage
\begin{lstbox}{Santa problem solution (elves)}
elfTex.wait()
mutex.wait()
    elves += 1
    if elves == 3:
        santaSem.signal()
    else
        elfTex.signal()
mutex.signal()

getHelp()

mutex.wait()
    elves -= 1
    if elves == 0:
       elfTex.signal()
mutex.signal()
\end{lstbox}

The first two elves release {\tt elfTex} at the same time they release
the {\tt mutex}, but the last elf holds {\tt elfTex}, barring other
elves from entering until all three elves have invoked {\tt getHelp}.

The last elf to leave releases {\tt elfTex}, allowing the
next batch of elves to enter.

\newpage
\section{Building H$_2$O}
\label{water}

This problem has been a staple of the Operating Systems class
at U.C. Berkeley for at least a decade.  It seems to be based on
an exercise in Andrews's {\em Concurrent Programming} \cite{andrews}.

There are two kinds of threads, oxygen and hydrogen.  In order
to assemble these threads into water molecules, we have to
create a barrier that makes each thread wait until a
complete molecule is ready to proceed.

As each thread passes the barrier, it should invoke
{\tt bond}.  You must guarantee that all the threads
from one molecule invoke {\tt bond} before any of the threads
from the next molecule do.

In other words:

\begin{itemize}

\item If an oxygen thread arrives at the barrier when no
hydrogen threads are present, it has to wait for two
hydrogen threads.

\item If a hydrogen thread arrives at the barrier when
no other threads are present, it has to wait for an
oxygen thread and another hydrogen thread.

\end{itemize}

We don't have to worry about matching the threads up explicitly; that
is, the threads do not necessarily know which other threads they are
paired up with.  The key is just that threads pass the barrier in
complete sets; thus, if we examine the sequence of threads that invoke
{\tt bond} and divide them into groups of three, each group should
contain one oxygen and two hydrogen threads.

\begin{puzzlebox}{Puzzle}
Write synchronization code for oxygen and hydrogen
molecules that enforces these constraints.


%% \clearemptydoublepage
%%% \subsec {H$_2$O hint}
\tcbsubtitle{Hint}

Here are the variables I used in my solution:

\begin{lstbox}{Water building hint}
mutex = Semaphore(1)
oxygen = 0
hydrogen = 0
barrier = Barrier(3)
oxyQueue = Semaphore(0)
hydroQueue = Semaphore(0)
\end{lstbox}

{\tt oxygen} and {\tt hydrogen} are counters, protected by {\tt mutex}.
{\tt barrier} is where each set of three threads meets after
invoking {\tt bond} and before allowing the next set of threads
to proceed.

{\tt oxyQueue} is the semaphore oxygen threads wait on;
{\tt hydroQueue} is the semaphore hydrogen threads wait on.
I am using the naming convention for queues, so
{\tt oxyQueue.wait()} means ``join the oxygen queue'' and
{\tt oxyQueue.signal()} means ``release an oxygen thread from
the queue.''
\end{puzzlebox}


%% \clearemptydoublepage
%%% \subsec {H$_2$O solution}
\sub{H$_2$O solution}

Initially {\tt hydroQueue} and {\tt oxyQueue} are locked.  When
an oxygen thread arrives it signals {\tt hydroQueue} twice,
allowing two hydrogens to proceed.  Then the oxygen thread waits
for the hydrogen threads to arrive.

\begin{lstbox}{Oxygen code}
mutex.wait()
oxygen += 1
if hydrogen >= 2:
    hydroQueue.signal(2)
    hydrogen -= 2
    oxyQueue.signal()
    oxygen -= 1
else:
    mutex.signal()

oxyQueue.wait()
bond()

barrier.wait()
mutex.signal()
\end{lstbox}

As each oxygen thread enters, it gets the mutex and checks the scoreboard.
If there are at least two hydrogen threads waiting, it signals two of
them and itself and then bonds.  If not, it releases the mutex and
waits.

After bonding, threads wait at the barrier until all three threads
have bonded, and then the oxygen thread releases the mutex.  Since
there is only one oxygen thread in each set of three, we are guaranteed
to signal {\tt mutex} once.

The code for hydrogen is similar:

\begin{lstbox}{Hydrogen code}
mutex.wait()
hydrogen += 1
if hydrogen >= 2 and oxygen >= 1:
    hydroQueue.signal(2)
    hydrogen -= 2
    oxyQueue.signal()
    oxygen -= 1
else:
    mutex.signal()

hydroQueue.wait()
bond()

barrier.wait()
\end{lstbox}

An unusual feature of this solution is that
the exit point of the mutex is ambiguous.  In
some cases, threads enter the mutex, update the counter, and exit the
mutex.  But when a thread arrives that forms a complete set, it has to
keep the mutex in order to bar subsequent threads until the current
set have invoked {\tt bond}.

After invoking {\tt bond}, the three threads wait at a barrier.
When the barrier opens, we know that all three threads have invoked
{\tt bond} and that one of them holds the mutex.  We don't know
{\em which} thread holds the mutex, but it doesn't matter as long
as only one of them releases it.  Since we know there is only one
oxygen thread, we make it do the work.

This might seem wrong, because until now it
has generally been true that a thread has to hold a lock in
order to release it.  But there is no rule that says that has
to be true.  This is one of those cases where it can be misleading
to think of a mutex as a token that threads acquire and release.


\blankpage
\section {River crossing problem}

This is from a problem set written by Anthony Joseph
at U.C. Berkeley, but I don't know if he is the original author.
It is similar to the H$_2$O problem in the sense that it is
a peculiar sort of barrier that only allows threads to pass
in certain combinations.

Somewhere near Redmond, Washington there is a rowboat that is used by
both Linux hackers and Microsoft employees (serfs) to cross a river.  The
ferry can hold exactly four people; it won't leave the shore with more
or fewer.  To guarantee the safety of the passengers, it is not
permissible to put one hacker in the boat with three serfs, or to
put one serf with three hackers.  Any other combination is safe.

As each thread boards the boat it should invoke a function
called {\tt board}.  You must guarantee that all four threads
from each boatload invoke {\tt board} before any of the threads
from the next boatload do.

After all four threads have invoked {\tt board}, exactly one of
them should call a function named {\tt rowBoat}, indicating
that that thread will take the oars.  It doesn't matter which thread
calls the function, as long as one does.

Don't worry about the direction of travel.  Assume we are
only interested in traffic going in one of the directions.


%% \clearemptydoublepage
%% \subsec {River crossing hint}
\begin{puzzlebox}{Hint}

Here are the variables I used in my solution:

\begin{lstbox}{River crossing hint}
barrier = Barrier(4)
mutex = Semaphore(1)
hackers = 0
serfs = 0
hackerQueue = Semaphore(0)
serfQueue = Semaphore(0)
local isCaptain = False
\end{lstbox}

{\tt hackers} and {\tt serfs} count the number of hackers
and serfs waiting to board.  Since they are both protected by
{\tt mutex}, we can check the condition of both variables without
worrying about an untimely update.  This is another example
of a scoreboard.

{\tt hackerQueue} and {\tt serfQueue} allow us to control the number
of hackers and serfs that pass.  The barrier
makes sure that all four threads have invoked
{\tt board} before the captain invokes {\tt rowBoat}.

{\tt isCaptain} is a local variable that
indicates which thread should invoke {\tt row}.
\end{puzzlebox}

%% \clearemptydoublepage
%% \subsec {River crossing solution}
\sub{River crossing solution}

The basic idea of this solution is that each arrival updates
one of the counters and then checks whether it makes a
full complement, either by being the fourth of its kind or
by completing a mixed pair of pairs.

I'll present the code for hackers; the serf code is
symmetric (except, of course, that it is 1000 times bigger,
full of bugs, and it contains an embedded web browser):

\begin{lstbox}{River crossing solution}
mutex.wait()
    hackers += 1
    if hackers == 4:
        hackerQueue.signal(4)                
	hackers = 0
	isCaptain = True
    elif hackers == 2 and serfs >= 2:
        hackerQueue.signal(2)                
        serfQueue.signal(2)                  
	serfs -= 2
	hackers = 0
	isCaptain = True
    else:
        mutex.signal()      # captain keeps the mutex

hackerQueue.wait()           

board()
barrier.wait()            

if isCaptain:
    rowBoat()
    mutex.signal()          # captain releases the mutex
\end{lstbox}

As each thread files through the mutual exclusion section, it
checks whether a complete crew is ready to board.  If so, it
signals the appropriate threads, declares itself captain, and
holds the mutex in order to bar additional threads until the
boat has sailed.

The barrier keeps track of how many threads have boarded.
When the last thread arrives, all threads proceed.
The captain invoked {\tt row} and then (finally) releases the mutex.


%% \clearemptydoublepage
\blankpage
\section{The roller coaster problem}

This problem is from Andrews's {\em Concurrent
Programming} \cite{andrews}, but he attributes it to J. S. Herman's
Master's thesis.

\begin {quotation}
Suppose there are $n$ passenger threads and a car thread.  The passengers
repeatedly wait to take rides in the car, which can hold $C$ passengers,
where $C<n$.  The car can go around the tracks only when it is full.
\end{quotation}

Here are some additional details:

\begin{itemize}

\item Passengers should invoke {\tt board} and {\tt unboard}.

\item The car should invoke {\tt load}, {\tt run} and {\tt unload}.

\item Passengers cannot board until the car has invoked {\tt load}

\item The car cannot depart until $C$ passengers have boarded.

\item Passengers cannot unboard until the car
has invoked {\tt unload}.

\end{itemize}

\begin{puzzlebox}{Puzzle}
Write code for the passengers and car that enforces these
constraints.


%% \clearemptydoublepage
%%% \subsec{Roller Coaster hint}
\tcbsubtitle{Hint}

\begin{lstbox}{Roller Coaster hint}
mutex = Semaphore(1)
mutex2 = Semaphore(1)
boarders = 0
unboarders = 0
boardQueue = Semaphore(0)
unboardQueue = Semaphore(0)
allAboard = Semaphore(0)
allAshore = Semaphore(0)
\end{lstbox}

{\tt mutex} protects {\tt passengers}, which counts the number of
passengers that have invoked {\tt boardCar}.  

Passengers wait on {\tt boardQueue} before boarding and
{\tt unboardQueue} before unboarding.  {\tt allAboard}
indicates that the car is full.
\end{puzzlebox}


%% \clearemptydoublepage
%%% \subsec{Roller Coaster solution}
\sub{Roller Coaster solution}

Here is my code for the car thread:

\begin{lstbox}{Roller Coaster solution (car)}
load()
boardQueue.signal(C)
allAboard.wait()

run()

unload()
unboardQueue.signal(C)
allAshore.wait()
\end{lstbox}

When the car arrives, it signals $C$ passengers,
then waits for the last one to signal {\tt allAboard}.
After it departs, it allows $C$ passengers to disembark,
then waits for {\tt allAshore}.

\begin{lstbox}{Roller Coaster solution (passenger)}
boardQueue.wait()
board()

mutex.wait()
   boarders += 1
   if boarders == C:
       allAboard.signal()
       boarders = 0
mutex.signal()

unboardQueue.wait()
unboard()

mutex2.wait()
   unboarders += 1
   if unboarders == C:
       allAshore.signal()
       unboarders = 0
mutex2.signal()
\end{lstbox}

Passengers wait for the car before boarding, naturally, and wait for
the car to stop before leaving.  The last passenger to board signals
the car and resets the passenger counter.


%% \clearemptydoublepage
\blankpage
\subsec{Multi-car Roller Coaster problem}

This solution does not generalize to the case where there is more
than one car.  In order to do that, we have to satisfy some additional
constraints:

\begin{itemize}

\item Only one car can be boarding at a time.

\item Multiple cars can be on the track concurrently.

\item Since cars can't pass each other, they have to unload
in the same order they boarded.

\item All the threads from one carload must disembark before
any of the threads from subsequent carloads.

\end{itemize}

\begin{puzzlebox}{Puzzle}
Modify the previous solution to handle the additional constraints.
You can assume that there are $m$ cars, and that
each car has a local variable named {\tt i}
that contains an identifier between 0 and $m-1$.


%% \clearemptydoublepage
%%% \subsec{Multi-car Roller Coaster hint}
\tcbsubtitle{Hint}

I used two lists of semaphores to keep the cars in order.  One
represents the loading area and one represents the unloading area.
Each list contains one semaphore for each car.
At any time, only one semaphore in each
list is unlocked, so that enforces the order threads can
load and unload.
Initially, only the semaphores for Car 0 are unlocked.
As each car enters the
loading (or unloading) it waits on its own semaphore; as it leaves it
signals the next car in line.

\begin{lstbox}{Multi-car Roller Coaster hint}
loadingArea = [Semaphore(0) for i in range(m)]
loadingArea[1].signal()
unloadingArea = [Semaphore(0) for i in range(m)]
unloadingArea[1].signal()
\end{lstbox}

The function {\tt next} computes the identifier of the next
car in the sequence (wrapping around from $m-1$ to 0):

\begin{lstbox}{Implementation of {\tt next}}
def next(i):
    return (i + 1) % m
\end{lstbox}
\end{puzzlebox}


%% \clearemptydoublepage
%%% \subsec{Multi-car Roller Coaster solution}
\sub{Multi-car Roller Coaster solution}

Here is the modified code for the cars:

\begin{lstbox}{Multi-car Roller Coaster solution (car)}
loadingArea[i].wait()
load()
boardQueue.signal(C)
allAboard.wait()
loadingArea[next(i)].signal()

run()

unloadingArea[i].wait()
unload()
unboardQueue.signal(C)
allAshore.wait()
unloadingArea[next(i)].signal()
\end{lstbox}

The code for the passengers is unchanged.

