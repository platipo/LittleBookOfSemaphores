Most undergraduate Operating Systems textbooks have a module on
Synchronization, which usually presents a set of primitives
(mutexes, semaphores, monitors, and sometimes condition variables),
and classical problems like readers-writers and
producers-consumers.

When I took the Operating Systems class at Berkeley, and taught it at
Colby College, I got the impression that most students were able to
understand the solutions to these problems, but few would have been
able to produce them, or solve similar problems.

One reason students don't understand this material deeply is that
it takes more time, and more practice, than most classes can spare.
Synchronization is just one of the modules competing for space in
an Operating Systems class, and I'm not sure I can argue that it is
the most important.  But I do think it is one of the most challenging,
interesting, and (done right) fun.

I wrote the first edition this book with the goal of identifying
synchronization idioms and patterns that could be 
understood in isolation and then assembled to solve complex problems.
This was a challenge, because synchronization code doesn't compose
well; as the number of components increases, the number of interactions
grows unmanageably.

Nevertheless, I found patterns in the solutions I saw, and discovered
at least some systematic approaches to assembling solutions that are
demonstrably correct.

I had a chance to test this approach when I taught Operating Systems
at Wellesley College.  I used the first edition of {\em The Little
Book of Semaphores} along with one of the standard textbooks, and I
taught Synchronization as a concurrent thread for the duration of the
course.  Each week I gave the students a few pages from the book,
ending with a puzzle, and sometimes a hint.  I told them not to look
at the hint unless they were stumped.

I also gave them some tools for testing their solutions: a small
magnetic whiteboard where they could write code, and a stack of
magnets to represent the threads executing the code.

The results were dramatic.  Given more time to absorb the material,
students demonstrated a depth of understanding I had not seen before.
More importantly, most of them were able to solve most of the puzzles.
In some cases they reinvented classical solutions; in other cases
they found creative new approaches.

When I moved to Olin College, I took the next step and created a
half-class, called Synchronization, which covered
{\em The Little Book of Semaphores} and also the implementation of
synchronization primitives in x86 Assembly Language, POSIX, and
Python.

The students who took the class
helped me find errors in the first edition and several
of them contributed solutions that were better than mine.  At the
end of the semester, I asked each of them to write a new,
original problem (preferably with a solution).  I have added their
contributions to the second edition.

Also since the first edition appeared, Kenneth Reek presented the
article ``Design Patterns for Semaphores'' at the ACM Special Interest
Group for Computer Science Education.  He presents a problem, which
I have cast as the Sushi Bar Problem, and two solutions that demonstrate
patterns he calls ``Pass the baton'' and ``I'll do it for you.''
Once I came to appreciate these patterns, I was able to apply them
to some of the problems from the first edition and produce solutions
that I think are better.

One other change in the second edition is the syntax.  After I wrote
the first edition, I learned Python, which is not only a great
programming language; it also makes a great pseudocode language.  So I
switched from the C-like syntax in the first edition to syntax
that is pretty close to executable Python\footnote{The primary
difference is that I sometimes use indentation to indicate code that
is protected by a mutex, which would cause syntax errors in Python.}.
In fact, I have written a simulator that can execute many of the
solutions in this book.

Readers who are not familiar with Python will (I hope) find it mostly
obvious.  In cases where I use a Python-specific feature, I explain the
syntax and what it means.  I hope that these changes make the book
more readable.

The pagination of this book might seem peculiar, but there is a method
to my whitespace.  After each puzzle, I leave enough space that the
hint appears on the next sheet of paper and the solution on the next
sheet after that.  When I use this book in my class, I hand it out a
few pages at a time, and students collect them in a binder.  My
pagination system makes it possible to hand out a problem without
giving away the hint or the solution.  Sometimes I fold and staple the
hint and hand it out along with the problem so that students can
decide whether and when to look at the hint.  If you print the book
single-sided, you can discard the blank pages and the system still
works.

This is a Free Book, which means that anyone is welcome to read,
copy, modify and redistribute it, subject to the restrictions of the
license.  I hope that people
will find this book useful, but I also hope they will help continue
to develop it by sending in corrections, suggestions, and additional
material.  Thanks!

\vspace{0.3in}

\noindent Allen B. Downey \\
\noindent Needham, MA \\
\noindent June 1, 2005 \\


\section*{Contributor's list}

The following are some of the people who have contributed to this
book:

\begin{itemize}

\item Many of the problems in this book are variations of classical
problems that appeared first in technical articles and then in textbooks.
Whenever I know the origin of a problem or solution, I acknowledge it
in the text.

\item I also thank the students at Wellesley College who worked with
the first edition of the book, and the students at Olin College who
worked with the second edition.

\item Se Won sent in a small but important correction in my presentation
of Tanenbaum's solution to the Dining Philosophers Problem.

\item Daniel Zingaro punched a hole in the Dancer's problem, which
provoked me to rewrite that section.  I can only hope that it makes more
sense now.  Daniel also pointed out an error in a previous version of
my solution to the H$_2$O problem, and then wrote back a year later
with some typos.

\item Thomas Hansen found a typo in the Cigarette smokers problem.

\item Pascal R\"{u}tten pointed out several typos, including my embarrassing
misspelling of Edsger Dijkstra.

\item Marcelo Johann pointed out an error in my solution to the
Dining Savages problem, and fixed it!

\item Roger Shipman sent a whole passel of corrections as well as
an interesting variation on the Barrier problem.

\item Jon Cass pointed out an omission in the discussion of dining
philosophers.

\item Krzysztof Ko\'{s}ciuszkiewicz sent in several corrections, including
a missing line in the Fifo class definition.

\item Fritz Vaandrager at the Radboud University Nijmegen in the
Netherlands and his students Marc Schoolderman, Manuel Stampe and Lars
Lockefeer used a tool called UPPAAL to check several of the solutions
in this book and found errors in my solutions to the Room Party problem
and the Modus Hall problem.

\item Eric Gorr pointed out an explanation in Chapter 3 that was
not exactly right.

\item Jouni Lepp\"{a}j\"{a}rvi helped clarify the origins of semaphores.

\item Christoph Bartoschek found an error in a solution to
the exclusive dance problem.

\item Eus found a typo in Chapter 3.

\item Tak-Shing Chan found an out-of-bounds error in {\tt counter\_mutex.c}.

\item Roman V. Kiseliov made several suggestions for improving
the appearance of the book, and helped me with some \LaTeX~issues.

\item Alejandro C\'{e}spedes is working on the Spanish translation of this
book and found some typos.

\item Erich Nahum found a problem in my adaptation of Kenneth Reek's
  solution to the Sushi Bar Problem.

\item Martin Storsj\"{o} sent a correction to the generalized smokers problem.

\item Cris Hawkins pointed out an unused variable.

\item Adolfo Di Mare found the missing ``and''.

\item Simon Ellis found a typo.

\item Benjamin Nash found a typo, an error in one solution, and
a malfeature in another.

\item Alejandro Pulver found a problem with the Barbershop solution.

\end{itemize}

% endcontrib

